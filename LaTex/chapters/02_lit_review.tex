%The existing literature on the use of repression by a state have mostly focused on the impact of  of domestic factors such as institutions and economic growth \citep{davenport_state_2007-1}.
Existing literature on repressions has focused mostly on their consequences and legacies \citep{rozenas_political_2017, lupu_legacy_2017, zhukov_stalins_2018}. As far as the strategic use of repressions by the state is studied, it is usually in relation to domestic factors such as institutions and economic shocks \citep{davenport_state_2007, greitens_dictators_2016, blaydes_state_2018} with less attention being given to external forces. 

\citet{davenport_state_2007} finds that democracy is correlated with lower levels of repression. However, it is mainly free electoral competition rather than constraints on power of the executive that accounts for this negative effect.  
\citet{greitens_dictators_2016} links the severity of repression to the threat from dictator's inner circle. A dictator who fears that he would be deposed in a coup rather than in popular uprising will fragment their coercive apparatus in order to weaken the power of a potential challenger from within. The weakened secret police will be, according to Greitens, more likely to use violence since it fails to identify the transgressors and cannot effectively deter dissent. 
Other scholars see repression as a substitute for co-option \citep{wintrobe_political_1998, svolik_politics_2012}. Instead relying on the threat of persecution, an authoritarian ruler might buy the loyalty of the population by distributing rents to the supporters of the regime usually through the party apparatus. Negative economic shocks can then increase repression since the rents are no longer available.  \citet{blaydes_state_2018} 
illustrates this on the case of Iraq under Hussein where lower oil prices 

Furthermore, \citet{blaydes_state_2018} presents a theory attempting to explain different levels of repression across ethnic minorities within a country. She argues that the nature of repression  depends on the legibility of the ethnic group to the state.\footnote{The term legibility in this context means ability of a state to identify individuals in a given population and gather information on them.}
%(i.e. ability to gather information on its population) 
 Since the state coercive institutions cannot reliably identify transgressors in  less legible population (because of, for example, greater cultural and linguistic distance), they will more tend to resort to collective punishment. The logic behind this is that the members of the group will police its members to avoid collective punishment. 
%argues that repression can be a subsitute for economic rents.  

Our research also contributes to the literature studying factors that influence position of a state towards its ethnic minorities and under what conditions conflict is likely to occur. Size and and distribution of ethnic groups have been emphasized. Several scholars pointed out that
states with large number of ethnic groups are more likely to violently repress calls for autonomy or secession to discourage other ethnic minorities from making similar demands in the future \citep{evera_hypotheses_1994, toft_geography_2005,walter_reputation_2009}. 
% On the other hand, binational countries such as former Czechoslovakia  
Furthermore,  \citet{toft_geography_2005} argues that geographically concentrated groups tend to view their ethnic homeland as indivisible and non-negotiable issue which increases the likelihood of violent conflict. However, these approaches fail to explain changes in state's attitudes to minorities over short periods of time when the size and distribution of ethnic groups remains roughly constant. 
% možná tam hoď Votes and Violence od Wiliknsona

More recently, the role of international factors have received greater attention.  \citet{butt_secession_2017}  argues that response of a state to a secessionist movement depends on external security environment and outside actors. Specifically, a state located in a war-prone region is more likely to suppress demands for secession because loss of territory and population would make it vulnerable  to a potential future attack. Furthermore, a state responds with more violence if a separatist movement receives a support from an external power since the outside assistance makes the secessionists stronger. In addition to these strategic reasons, Butt emphasizes that  receiving external support incites  a strong feeling of betrayal to the central state.   
%\citet{svolik_politics_2012} sees repression as one of the tools of the authorian regime to keep conrol. Co-option and repression. Whereas the co-option is costly, repression requires greater power to the military. This might theaten the dictator's rule. However, this does not address different levels of repressions applied to different ethnic groups. 

\citet{mylonas_politics_2013} puts forward a theory explaining how geopolitical relations influence the attitude of a state towards its minorities. The model features an ethnic minority living within a host state and  an external power. Moreover, Mylonas distinguishes between hosts states with revisionist foreign policy which want change the international status quo (e.g. because they gained power or lost territory in the past) and host states that prefer the current international order. 

The predictions of the theory are summarized in the table \ref{tab:mylonas}. First, if a minority group is not supported by any external power, the theory predicts that the host state will pursue policy of assimilation towards the group to \enquote{immunize} it from possible future  agitation  of external powers. Second, if an ethnic minority is supported by geopolitical ally then accommodation is likely since more repressive policies towards the minority could jeopardize the alliance. Third, theory predicts assimilation if a minority is supported by an geopolitical enemy and a host state pursues non-revisionist foreign policy because exclusionary policies could trigger new hostilities threatening the status quo. 
Finally, support of an external enemy combined with revisionist foreign policy will likely lead to exclusion of a given ethnic group since it is view as a potential \enquote{fifth column} of the external power. 

% Please add the following required packages to your document preamble:
% \usepackage{multirow}

\begin{table}[]
{\setstretch{2}
\begin{tabular}{llccc}
                                                                                                                     &                                           & \multicolumn{3}{c}{\textbf{External Power Support}}                                                                           \\ \cline{3-5} 
                                                                                                                     & \multicolumn{1}{l|}{}                     & \multicolumn{2}{c|}{\textbf{Yes}}                                        & \multicolumn{1}{c|}{\textbf{No}}                   \\ \cline{3-5} 
                                                                                                                     & \multicolumn{1}{l|}{}                     & \multicolumn{2}{c|}{Interstate Relations}                                & \multicolumn{1}{c|}{\multirow{4}{*}{Assimilation}} \\ \cline{3-4}
                                                                                                                     & \multicolumn{1}{l|}{}                     & \multicolumn{1}{c|}{\textbf{Ally}} & \multicolumn{1}{c|}{\textbf{Enemy}} & \multicolumn{1}{c|}{}                              \\ \cline{2-4}
\multicolumn{1}{l|}{\multirow{2}{*}{\textbf{\begin{tabular}[c]{@{}l@{}}Host's State\\ Foreign Policy\end{tabular}}}} & \multicolumn{1}{l|}{\textbf{Revisionist}} & \multicolumn{1}{c|}{Accommodation} & \multicolumn{1}{c|}{Exclusion}      & \multicolumn{1}{c|}{}                              \\ \cline{2-4}
\multicolumn{1}{l|}{}                                                                                                & \multicolumn{1}{l|}{\textbf{Status Quo}}  & \multicolumn{1}{c|}{Accommodation} & \multicolumn{1}{c|}{Assimilation}   & \multicolumn{1}{c|}{}                              \\ \cline{2-5} 
\end{tabular}
\caption{\label{tab:mylonas}Theoretical predictions of \citet{mylonas_politics_2013}}
\centering}
\end{table}



%In our study, we focus mainly on case when ex
Although in Mylonas' theory  external patron of a minority can be any state, 
we will considers only the  case of states whose political elites have ethnic ties to the given minority.  
Ethnic identity by its nature creates certain affinity even for co-ethnics outside given country. Empirically, the states are more likely to intervene in support of a 
ethnic minority to which they have ethnic ties \citet{saideman_ties_2001, saideman_discrimination_2002}. 
%We can thus consider that 
Host state can thus perceive certain implicit support from an external power based only on  the ethnic ties with the minority even it is not providing any real assistance. 
%We would argue
%Arguably, the ethnic ties can be more relevant than 
%Can
%This will be only strenghened 
%This will 
%Host states are well aware of the ethnic ties and 
%they can perceive certain implict
%they will likely take them into account. 

%Moreover, there are many cases of 
%Therefore we 
%The host are aware of these ties and thus 
% if we apply this to our case 

We can apply the theory to our case. First, the Soviet Union was arguably a revisionist state. The Bolsheviks had to accept large losses of territory under the Treaty of Brest-Litovsk in 1918 so that they could focus on fighting in Russian Civil War. Thus, the USSR would certainly prefer to change the international status quo. Second, the German-Soviet relations were neutral or even  friendly prior to 1933 but turned hostile after the rise of Hitler (described in greater detail in subsection \ref{subsec:relations}). Therefore the theory predicts that the policy 


%We can apply the theory to our case. First, the Soviet Union was arguably a revisionist state. The Bolsheviks had to accept large losses of territory under the Treaty of Brest-Litovsk in 1918 so that they could focus on fighting in Russian Civil War. Thus, the USSR would certainly prefer to change the international status quo. Second, the German-Soviet relations were neutral of even moderately friendly prior to 1933 but turned hostile after the rise of Hitler (described in greater detail in subsection \ref{subsec:relations}). Therefore the theory predicts that the policy of the Soviet state towards the German minority should change from accommodation to exclusion. \citet[p. 22]{mylonas_politics_2013} defines exclusion as \enquote{policies that aim at the physical removal of a non-core group from the host state.} The Soviet political repression which usually featured either outright execution or a term in a labor camp in the Far East of the country fits this description well. Thus, the theoretical expectation is that  repressions of Soviet Germans relative to other minorities should increase after 1933. 

%Beyond these strategic concerns there is even more emotional factor that can play a role. An ethnic groups can be 

Our main contribution is empirical test of this theory using a credible identification strategy. Most studies presented in this review test their hypotheses only by qualitative comparison of selected cases. Quantitative research usually involve only cross-sectional regressions based on categorical dependent variables.    

For example, \citet{mylonas_politics_2013} tests his theory with data on the post-World War I Balkans where the nation-building policies (categorized into 3 groups: accommodation, assimilation and exclusion)  toward  90 ethnic groups are a dependent variable and information on their support by external powers is an explanatory variables (together with other control variables). However, the results of the cross-sectional regression, used in the study, might easily be biased due to omitted variables or reverse causality and we believe that our approach offers cleaner identification.

%According to \citet{blaydes_state_2018}, a state will resort to collective punishment (based on ethnicity, religion or community membership) if it faces environment with highly asymmetric information in which it cannot identify the likely transgressors. The logic behind this is that the members of the community will police its members to avoid collective punishment. 


\citet{mcnamee_demographic_2019} is methodologically and thematically closet study to ours. They analyze how the 1958 split in Soviet-China relations affected the demographic composition of the population in the Soviet-Chinese border regions.
Using difference-indifference strategy, they find that, after the split,  both states supported expulsions  of the minority group and sponsored immigration of the majority group but only in border regions without significant natural boundary (e.g. high mountains). They conclude that the states use demographic engineering as a way to protect their vulnerable border against a hostile power. Nonetheless, as McNamee and Zhang only measure the ethnic composition of the regions they cannot unambiguously identify expulsions as the main culprit since other factors could plausibly affect voluntary migration. %Moreover, the dynamics of repressions and 


%Hod tam koyamu, Zhurkavskayu a doten-snitken