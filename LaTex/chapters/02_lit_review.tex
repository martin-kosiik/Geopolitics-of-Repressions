%The existing literature on the use of repression by a state have mostly focused on the impact of  of domestic factors such as institutions and economic growth \citep{davenport_state_2007-1}.
Existing literature on repressions has focused mostly on their consequences and legacies \citep{rozenas_political_2017, lupu_legacy_2017, zhukov_stalins_2018}. As far as the strategic use of repressions by the state is studied, it is usually in relation to domestic factors such as institutions and economic shocks \citep{davenport_state_2007-1, greitens_dictators_2016, blaydes_state_2018} with less attention being given to external forces. 

As was mentioned, \citet{mylonas_politics_2013} proposes a theory how of geopolitical relations influence the attitude of a state towards its minorities. He also tests his theory with data on the post-World War I Balkans where the nation-building policies (categorized into 3 groups: accommodation, assimilation and exclusion)  toward  90 ethnic groups are a dependent variable and information on their support by external powers is an explanatory variables (together with other control variables). However, the results of the cross-sectional regression, used in the study, might easily be biased due to omitted variables or reverse causality and we believe that our approach offers cleaner identification.

According to \citet{blaydes_state_2018}, a state will resort to collective punishment (based on ethnicity, religion or community membership) if it faces environment with highly asymmetric information in which it cannot identify the likely transgressors. The logic behind this is that the members of the community will police its members to avoid collective punishment. 

\citet{mcnamee_demographic_nodate} is methodologically and thematically closet study to ours. They analyze how the 1958 split in Soviet-China relations affected the demographic composition of the population in the Soviet-Chinese border regions.
Using difference-indifference strategy, they find that, after the split,  both states supported expulsions  of the minority group and sponsored immigration of the majority group but only in border regions without significant natural boundary (e.g. mountains). They conclude that the states use demographic engineering as a way to protect their vulnerable border against a hostile power. 
