\subsection{Difference-in-differences} \label{subsec:methodology_did}
We employ difference-in-differences design to estimate the effect of the changing geopolitical relations on repressions of Germans in the USSR. 
Our main specification is the dynamic difference-in-differences model:
\begin{equation}
 \log\left(1 + y_{it}\right) = \sum_{k= 1922}^{1960} \beta_k \, \text{German}_{i} \cdot \text{Year}_{t}^k +  \lambda_t + a_i +  \text{Relations}_{it}   + u_{it},
 \label{eq:dynamic_did}
\end{equation}
%+  E_i \cdot t \:  + E_i \cdot t^2 \: 
 %+ \sum_{j= 1}^5 \omega_j \, german_{it} \cdot post_{it - k}    

where $y_{it}$ is number of arrests of people with ethnicity $i$ in year $t$ (from 1921 to 1960), $\lambda$ is year fixed effect, $a$ is ethnicity fixed effect (both captured by respective dummy variables) and  $\text{Year}_{t}^k$ are dummy variables that equals 1 if its year $k$ equals to $t$ and 0 otherwise.
%(except for 1939 which is equal to 1 if $t \geq 1959$ and zero otherwise). The coefficients of interest are $\beta_k$. 
Prior to  1933 they capture the lead (anticipatory) effects  used to test if pre-treatment trends are parallel. After 1933 they capture the dynamic lagged effects. The only omitted year is 1921 which thus serves as a baseline against which other coefficients are compared. 
%The 5 coefficients $\omega_j$ capture the potential lagged effects (extending from 1934 to 1938), whereas the 3 coefficients capture the lead (anticipatory) effects (from 1930 to 1932) used to test pre-treatment parallel trends. 
The time-varying vector $\text{Relations}_{it}$ is set a of dummy variables that capture other major changes in relations of a state with core group $i$ with the USSR during this period. The list of these changes is provided in appendix in the table \ref{tab:sc_weights_robustness}.

In additional specifications used to test sensitivity of results, we also include  the terms $ E_i \cdot t$ and $ E_i \cdot t^2$  that can capture  ethnicity-specific time trends. 
This allows for a certain  deviation from the parallel trends if it can be described by the specified (quadratic or linear) function \citep[chapter 5]{angrist_mostly_2009}. 
%If the inclusion of these terms does not  significantly  change the coefficients, it would suggest that our results are not driven by spurious correlation.
However, a potential issue with the ethnicity-specific time trends is that they could absorb part of the treatment effect if it increases over time \citep{meer_effects_2016} which is why we do not include them in our default specification. 
 We apply logarithmic transformation on $y_{it}$ since it better fits the data.  We use $\log\left(1 + y_{it}\right)$ because some observations (although not many) have $y = 0$. As discussed in \citet[p. 193]{wooldridge_introductory_2015},  the 
 %percentage change interpretation
 properties of standard logarithmic transformation
 are usually  closely preserved (except for changes beginning at 0 which are not of great interest to us).   

For continuous variables, log-linear models have straightforward percentage change interpretation, i.e. for variable $x_j$ with corresponding coefficient $\gamma_j$, the estimated percentage change in $y$ for small change in $x_j$ is equal to $100 \cdot \gamma_j$. However, as \citet{halvorsen_interpretation_1980} point out, this interpretation is not correct for dummy variables. Fortunately, \citet{kennedy_estimation_1981} derives the following unbiased estimator for percentage change effect $p_j$ of a dummy variable $x_j$:
\begin{equation}
\hat p_j = 100 \cdot \left(\exp\left\{\hat\beta_j - \frac{1}{2} \hat V\left(\hat\beta_j\right)\right\} - 1 \right)
 \label{eq:dummy_vars}
\end{equation}
where $\hat\beta_j$ is the estimated coefficient on $x_j$ and  $\hat V\left(\hat\beta_j\right)$ is an estimate of variance of $\hat\beta_j$. 
Since our main coefficients of interest are dummy variables, we will use this formula to obtain estimate of percentage change effect.


%While the equation \ref{eq:dynamic_did} gives us information on the changes in the effect, it might be useful to have a single estimate of an average effect for different phases of the German-Soviet relations. For this reason, we also estimate a difference-in-differences model given by the following equation:
%The simple difference-in-differences can be expressed as
%\begin{equation}
%\begin{aligned}
% \log\left(1 + y_{it}\right) = 
% \lambda_t + a_i +  E_i \cdot t \:  + E_i \cdot t^2 \: +
%  \alpha \, \text{German}_{i} \cdot \text{PreTreat}_{t}   \\ % &\quad
%  + \beta \, \text{German}_{i} \cdot \text{Hostility}_{t} +
% \gamma \, \text{German}_{i} \cdot \text{Pact}_{t}  \\
% + \delta \, \text{German}_{i} \cdot \text{War}_{t} +
% \eta \, \text{German}_{i} \cdot \text{PostWar}_{t}  + u_{it}
% \label{eq:simple_did}
% \end{aligned}
%\end{equation}


%where the variables $\text{PreTreat}_{t}$, $\text{Hostility}_{t}$, $\text{Pact}_{t}$, $\text{War}_{t}$, and $\text{PostWar}_{t}$  are dummies that equal 1 if the year $t$ is within the given period. We define  $\text{PreTreat}_{t}$ to be from 1930 to 1932, $\text{Hostility}_{t}$ from 1933 to 1939, $\text{Pact}_{t}$ in 1940, $\text{War}_{t}$ from 1941 to 1944, and $\text{PostWar}_{t}$ from 1944 onwards.\footnote{We determined the phases based on what fraction of a given year was in the phase. Therefore even though the  World War 2  ended in Europe in May 1945, we  classify the year 1945 as being in the post-war period since that was true for greater fraction of the year. Similar logic applies to year 1939 as well. }
%All other terms are defined exactly the same as in the dynamic model in equation \ref{eq:dynamic_did}.
%However, the treatment status of some minorities changed in this period as they geopolitical relations 


The identifying assumption for the specifications \ref{eq:dynamic_did} is that the number of arrest of Germans after 1933 would go in parallel to arrests of other minorities in the absence shock to German-Soviet relations conditional on our control variables.\footnote{In the presence of linear ethnicity-specific time trends, the identifying assumption  is the parallel growth of the outcome variable of control and treatment group. This is equivalent to the parallel trends for the first differences of the outcome \citep{mora_alternative_2019}. } Although we cannot test this assumption, we can test whether the trends were parallel prior to 1933 (pre-treatment) which could increase our confidence that they were parallel after 1933 too. This can be testing if the coefficients $\beta_k$ on the lead effects are significantly different from zero.  

As \citet{bertrand_how_2004} show, the usual standard errors  are downward-biased for most difference-in-differences regressions since they do not account for the serial correlation within the units of interests (states, countries etc.). A common solution to this problem is to estimate standard errors using robust covariance matrix that allows for clustering (i.e. cluster-robust standard errors). However for small number of groups (generally less than 40), the cluster-robust standard errors are downward-biased and not reliable \citet[chapter 8]{angrist_mostly_2009}.
%suggest taking the maximum of cluster-robust as a simple rule of thumb to avoid gross misjudgements in precision. 
Possible solutions are cluster bootstrapping \citep{cameron_bootstrap-based_2008, cameron_practitioners_2015} and  using $t$-distribution with $G- K$ degrees of freedom (where $G$ is number of clusters and $K$ number of parameters) rather than the standard Normal distribution \citep{mccaffrey_bias_2002, imbens_robust_2016}.
Since we have small number of groups we use the generalization of \citet{mccaffrey_bias_2002} correction to models with arbitrary sets of fixed effects by \citet{pustejovsky_small-sample_2018}.


\subsection{Synthetic Control Method} \label{subsec:sc_methodology}
%However, the parallel trends assumption required for unbiasedness of difference-in-differences can be too restri for .
However, difference-in-differences makes a fairly strong assumption regarding the absence of time-varying individual-specific heterogeneity which might be unrealistic in some empirical settings.
The synthetic control method has been increasingly applied in the economic literature to overcome this issue \citep{abadie_economic_2003, abadie_synthetic_2010, billmeier_assessing_2013, cavallo_catastrophic_2013}. The method works by constructing a synthetic version of the treated unit from the control units  based on matching of pre-treatment variables. The outcomes of synthetic control are then compared to the actual outcomes  to estimate the treatment effect.
Unlike difference-in-differences, the synthetic control method allows for the presence of certain time-varying unit-specific confounders although only if they can be  captured by the factor model in the equation \ref{factor_model}.
%We 
%the parallel trends assumption required for unbiasedness of difference-in-differences can be too restri for .



%Moreover, difference-in-differences often extrapolate data beyond the region of common support. To address these potential issues, we use the synthetic control method which relaxes some of the difference-in-differences assumptions and thus provides us with an useful robustness check. 
% The synthetic control method has been applies to many topics. 
% \citep{abadie_economic_2003, abadie_synthetic_2010} 

Let $Y_{it}$ be the outcome of a unit $i$ at time $t$ with $i = 1$ being the treated group and $i \in \{ 2, \dots, J + 1 \}$ untreated units (we will also call them donor units).  We
denote $D_{1t}$ as the treatment dummy, i.e. variable that equals 1 if $i = 1$ and $t > T_0$ and 0 otherwise (with $T_0$ being the start of the treatment). 
Let be $Y_{1t}^N$ be a counterfactual outcome for the treated unit at time $t > T_0$ in the absence of treatment. 
The observed outcome $Y_{it}$ is then assumed to be  sum of the counterfactual outcome $Y_{1t}^N$ and the effect of treatment at time $t$, $\alpha_{1t}$, i.e.:
\begin{equation}
    Y_{1t} = Y_{1t}^N + \alpha_{1t} \, D_{1t}.
\end{equation}
The synthetic control method also assumes that $Y_{1t}^N$  can be expressed by the following factor model:
\begin{equation} \label{factor_model}
   Y_{1t}^N = \delta_t + \boldsymbol{\theta}_t \boldsymbol{Z}_i +
   \boldsymbol{\lambda}_t \boldsymbol{\mu}_i + \epsilon_{it},
\end{equation}
where is $\delta_t$ an unknown common factor with constant factor
loadings across units, $\boldsymbol{Z}_i$ is a
$(1 \times r)$ vector of observed time-invariant covariates (unaffected by the treatment),  $\boldsymbol{\theta}_t$ is a $(1 \times r)$ vector of
unknown parameters, $\boldsymbol{\lambda}_t$ is a $(1 \times F)$ vector of unobserved time-varying factors, $\boldsymbol{\mu}_i$ is an $(F \times 1)$ vector of unknown factor loadings
and $\epsilon_{it}$ is the error term with zero mean.

Notice that if $\boldsymbol{\lambda}_t$ is constant  for all $t$, we get the traditional
difference-in-differences with time and unit-specific fixed effects. 
The synthetic control method thus offers more general model that, in contrast to difference-in-differences, allows the unobserved confounders vary with time.

%Unlike difference-in-differences,  the synthetic control method  allows for unit-specific time trends as long as they can be captured by the factor model. 

We then try to estimate the counterfactual outcome $Y_{1t}^N$  by constructing a synthetic control group  as a convex combination of available comparison units (in our case other ethnic groups in the USSR) that most closely resembles the pre-treatment characteristics of the treated group
 (or more precisely, for which the average of its factor loadings $\boldsymbol{\mu}_i$ match the factor loadings of the treated unit  $\boldsymbol{\mu}_1$). 
 %The intuition be
 
 
 More formally, we choose a vector of weights $W = (w_2, \dots, w_J, w_{J+1})$ subject to $w_j \geq 0$ for $j = 2, \dots, J, J + 1$ and $w_2 +  \dots + w_J + w_{J+1} = 1$ that minimize $\left\Vert X_1 - X_0 W \right\Vert_V = \sqrt{\left(X_1 - X_0 W\right)^T V \left(X_1 - X_0 W\right)}$ where
  $X_1 = (Z_1, Y_1^{K_1}, \dots, Y_1^{K_L})$ is a $(k \times 1)$ vector of pre-treatment characteristics of the treated unit,  $Y^{K_l}$ are linear combinations of
pre-treatment outcomes, and %(analogously for $X_0$)
  $X_0$ is a $(k \times J)$ matrix with pre-treatment characteristics of untreated units analogous to $X_1$, 
 %$X_1$ is a vector of pre-treatment characteristics of the treated unit,
  $V$ is  a $(k \times k)$  matrix that weights the importance of different pre-treatment predictors in the minimization problem. 
  $V$ is usually chosen among symmetric and positive semidefinite matrices to minimize the  mean squared prediction error (MSPE) in the outcome in the pre-treatment period so that the $V$-weights would reflect the predictive power of the covariates. 
% matching variables that are chosen to 
 %$X_1 = (Z_1, Y_1^{K_1}, \dots, Y_1^{K_L})$ is a $(k \times 1)$ vector of pre-treatment characteristics of the treated unit and $k = r + L$ and $Y^{K_l}$ are combinations of
%pre-treatment outcomes (analogously for $X_0$). 
The effect of the treatment $\alpha_{1t}$ at time $t > T_0$ is then estimated as a difference between the outcome for the synthetic control and the treated unit, that is:
\begin{equation}
  \hat\alpha_{1t}  = Y_{1t} - \sum_{j = 2}^{J + 1} w_j^* Y_{jt},
  %\qquad t \in \{T_0 + 1, \dots, T \}.
\end{equation}
where $w^*_j$ are the estimated optimal weights. 

The synthetic control method, however, does not provide us with any standard errors that could measure the uncertainty in the estimates.
\citet{abadie_synthetic_2010} propose assessing significance  using  placebo tests and randomization inference. Synthetic control method is applied iteratively to every unit in the donor pool as if they were treated and the results of placebo test are then compared to the treated unit. If the estimated effect for the treated unit is much larger than the placebo effects, it implies that the effect is likely significant since 
such a result would not be likely under the null hypothesis of zero treatment effect.
%\citet{abadie_synthetic_2010} propose a procedure for obtaining Fisher exact  $p$-values based on randomization inference and placebo tests. 

%Therefore if the estimated effect for the treated unit is much larger than the placebo effects, it implies that such a result would not be likely under the null hypothesis of zero treatment effect and the $p$-value would then be accordingly small. 

However, \citet{abadie_synthetic_2010}  point out that placebo synthetic controls with poor pre-treatment fit do not provide good comparison for estimating rareness of large effect for a treated unit with a good  pre-treatment fit. They thus recommend excluding excluding placebo units with substantially higher pre-treatment MSPE relatively to the treated unit. 

Nevertheless, the choice of any level of the cutoff of pre-treatment MSPE is somewhat arbitrary. Better way to asses significance of results might be to compare the ratios of post/pre-treatment MSPE
%\footnote{More formally, the MSPE ratio is defined as $\text{MSPE ratio_j} = \sum_{t = T_0 + 1}^{T} Y_{jt} -  \hat{ Y_{jt}^N}$ } since
%Using  the ratios instead of the values of $\hat\alpha$ allows us to
%asses the significance of results reltice 
take into account different values of pre-treatment fit between the treated unit and the placebo tests. First, the  MSPE ratio for unit $j$ is defined as
\begin{equation} \label{eq:mspe_ratio}
\text{MSPE ratio}_j = \frac{\sum_{t = T_0 + 1}^{T} \left( Y_{jt} -  \hat Y_{jt}^N\right)^2}{\sum_{t = 1}^{T_0} \left( Y_{jt} -  \hat Y_{jt}^N\right)^2} 
\end{equation}
where $\hat Y_{jt}^N$ is estimated synthetic control for unit $j$ at time $t$. 
The  $p$-values can then be calculated as 
\begin{equation} \label{eq:p_ratio}
 p_1   = \frac{\sum_{j = 2}^{J + 1} \mathbbm{1}\left( \text{MSPE ratio}_{j} \geq \text{MSPE ratio}_{1} \right)}{J + 1}.
\end{equation}
where $\mathbbm{1}$ is an indicator function that equals 1 if the equation in its argument is true and 0 otherwise.  

%would suggest that the results are significant since these results would be less likely if there were no treatment effect 
%Large difference between the outcomes of synthetic control and actual data is can be indicative of large effect 
%High deviation of synthetic control from the actual 
% The randomization inference p-value is the proportion of times the placebo treatment effect was larger than the estimated treatment effect


%Large gaps for arrests of Germans relative to other ethnic groups would suggest that the results are significant since these results would be less likely if there were no treatment effect . 


