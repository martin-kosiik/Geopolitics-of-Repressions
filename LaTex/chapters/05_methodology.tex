\subsection{Difference-in-differences}
Our main specification is the dynamic difference-in-differences model:
\begin{equation}
 \log\left(1 + y_{it}\right) = \sum_{k= 1930}^{1939} \beta_k \, \text{German}_{i} \cdot \text{Year}_{t}^k + \lambda_t + a_i +  E_i \cdot t \:  + E_i \cdot t^2 \:    + u_{it}
 \label{eq:dynamic_did}
\end{equation}

 %+ \sum_{j= 1}^5 \omega_j \, german_{it} \cdot post_{it - k}    

where $y_{it}$ is number of arrests of people with ethnicity $i$ in year $t$, $\lambda$ is year fixed effect, $a$ is ethnicity fixed effect (both captured by respective dummy variables) and  $\text{Year}_{t}^k$ are dummy variables that equals 1 if its year $k$ equals to $t$ and 0 otherwise (except for 1939 which is equal to 1 if $t \geq 1939$ and zero otherwise) . The coefficients of interest are $\beta_k$. 
Prior to  1933 they capture the lead (anticipatory) effects  used to test if pre-treatment trends are parallel. After 1933 they capture the dynamic lagged effects.
%The 5 coefficients $\omega_j$ capture the potential lagged effects (extending from 1934 to 1938), whereas the 3 coefficients capture the lead (anticipatory) effects (from 1930 to 1932) used to test pre-treatment parallel trends. 
The $ E_i \cdot t$ and $ E_i \cdot t^2$  term capture the ethnicity specific quadratic time trends. The inclusion of this term should not significantly  change the coefficients, unless the results are driven by spurious correlation (\citealt{angrist_mostly_2009}). 

 We apply logarithmic transformation on $y_{it}$ since it better fits the data (more in the results below).  We use $\log\left(1 + y_{it}\right)$ because some observations (although not many) have $y = 0$. As discussed in \citet[p. 193]{wooldridge_introductory_2015},  the percentage change interpretation is usually  closely preserved (except for changes beginning at 0 which are not of interest to us).   

Our identifying assumption is that the number of arrest of Germans after 1933 would go in parallel to arrests of other minorities in the absence shock to German-Soviet relations conditional on our control variables (mainly the ethnicity specific time trends). Although we cannot test this assumption, we can test whether the trends were parallel prior to 1933 (pre-treatment) which could increase our confidence that they were parallel after 1933 too. This can be testing if the coefficients $\beta_k$ on the lead effects are significantly different from zero.  

As \citet{bertrand_how_2004} show, the usual standard errors  are downward-biased for most DiD regressions since they do not account for the serial correlation within the units of interests (states, countries etc.). A common solution to this problem is to estimate standard errors using robust covariance matrix that allows for clustering (i.e. cluster-robust standard errors). However for small number of groups (generally less than 40), the cluster-robust standard errors are downward-biased and not reliable. \citet[chapter 8]{angrist_mostly_2009} suggest taking the maximum of cluster-robust as a simple rule of thumb to avoid gross misjudgements in precision. More rigorous solutions are cluster bootstrapping \citep{cameron_bootstrap-based_2008, cameron_practitioners_2015} and  using $t$-distribution with $G- K$ degrees of freedom (where $G$ is number of clusters and $K$ number of parameters) rather than the standard Normal distribution \citep{mccaffrey_bias_2002, imbens_robust_2016}.
Since we have small number of groups we use the generalization of \citet{mccaffrey_bias_2002} correction to models with arbitrary sets of fixed effects by \citet{pustejovsky_small-sample_2018}.


\subsection{Synthetic Control Method}
However, the parallel trends assumption required for unbiasedness of difference-in-differences is often violated. Moreover, difference-in-differences often extrapolate data beyond the region of common support. To address these potential issues, we use the synthetic control method which relaxes some of the difference-in-differences assumptions and thus provides us with an useful robustness check. 
% The synthetic control method has been applies to many topics. 
% \citep{abadie_economic_2003, abadie_synthetic_2010} 

Let $Y_{it}$ be the outcome of a unit $i$ at time $t$ with $i = 1$ being the treated group.  We
denote $D_{1t}$ as the treatment dummy, i.e. variable that equals 1 if $i = 1$ and $T > T_0$ and 0 otherwise (with $T_0$ being the start of the treatment). 
Let be $Y_{1t}^N$ be a counterfactual outcome for the treated unit in the absence of treatment. The effect of treatment a time $t$, $\alpha_{1t}$ is assumed to be given as 
\begin{equation}
    Y_{1t} = Y_{1t}^N + \alpha_{1t} \, D_{1t}
\end{equation}
Furthermore, the synthetic control method assumes that $Y_{1t}^N$  can be expressed by the following factor model:
\begin{equation}
   Y_{1t}^N = \delta_t + \boldsymbol{\theta}_t \boldsymbol{Z}_i +
   \boldsymbol{\lambda}_t \boldsymbol{\mu}_i + \epsilon_{it}
\end{equation}
where is $\delta_t$ an unknown common factor with constant factor
loadings across units, $\boldsymbol{Z}_i$ is a
$(1 \times r)$ vector of observed time-invariant covariates (unaffected by the treatment),  $\boldsymbol{\theta}_t$ is a $(1 \times r)$ vector of
unknown parameters, $\boldsymbol{\lambda}_t$ is a $(1 \times F)$ vector of unobserved time-varying factors, $\boldsymbol{\mu}_i$ is an $(F \times 1)$ vector of unknown factor loadings
and $\epsilon_{it}$ is the error term with zero mean.

Notice that for constant  $\boldsymbol{\lambda}_t$ for all $t$ we get the traditional  difference-in-differences model. Unlike difference-in-differences,  the synthetic control method  allows for unit-specific time trends as long as they can be captured by the factor model. 

The synthetic control is  constructed as a convex combination of available comparison units (in our case other minorities in the USSR) that most closely resembles the pre-treatment characteristics of the treated group
 (or more precisely, for which the average of its factor loadings $\boldsymbol{\mu}_i$ match the factor loadings of the treated unit  $\boldsymbol{\mu}_1$). 
 More formally we choose weights $W = (w_2, \dots, w_J, w_{J+1})$ subject to $w_j \geq 0$ for $j = 1, \dots, J, J + 1$ and $w_2 +  \dots + w_J + w_{J+1} = 1$ that minimize $\left\Vert X_1 - X_0 W \right\Vert$ where $X_1 = (Z_1, Y_1^{K_1}, \dots, Y_1^{K_L})$ is a $(k \times 1)$ vector of pre-treatment characteristics of the treated unit and $k = r + L$ and $Y^{K_l}$ are combinations of
pre-treatment outcomes (analogously for $X_0$). The effect of the treatment at time $t$, $\alpha_{1t}$, is then estimated as a difference between the outcome for synthetic control and the treated unit, i.e.:
\begin{equation*}
  \hat\alpha_{1t}  = Y_{1t} - \sum_{k = 2}^{J + 1} w_j^* Y_{kt}
\end{equation*}

