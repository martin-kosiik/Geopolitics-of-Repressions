We used difference-in-differences and the synthetic control method to test how changing geopolitical relations between Soviet Union and Germany  affected repressions of Germans by the Soviet secret police. 
Both models suggest that up to 80\% of  repressions of Germans in the period from 1933 to 1960 might be attributable to the changes in geopolitical relations. 
However, this is rather an upper bound estimate since it assumes absence of any German-specific  post-1933 confounders not captured by our models. 
The increase in repressions of Germans is the highest and most significant in the years following the German invasions in 1941. 
%Both methods provide evidence that the war significantly increased the arrests of Germans.

%Our upper bound estimate is that around 80\% of the repression of Germans from 1933 to 1960 are attributable to the geopolitical relations.   
%Overall, the estimates imply that around 80\% of the repression are attributable to the  
%Specifically, the models estimate  rise in the average of arrests during the war even though are some year-to-year deviation. 

Furthermore, we find that the increased repression persist almost undiminished for nearly 10 year after the end of war when the security concerns are no longer present.
This  suggests that use of violence by the state might be largely driven by out-group 
hostility rather than the strategic considerations emphasized in the literature since after 1945 divided Germany did not posed a serious geopolitical threat anymore. 
The strong and long persistence of the hostile attitudes after the war could potentially help explain the phenomenon of conflict trap (i.e. why violence tends to reoccur in the same places). 
% hod tam persitence of conlift, conflict trap
% which remains to be explored by further research
However, our methods do not enable us to determine the underlining mechanism. It could be the bias of  rank and file officers of the secret police or some  directives from the top. %Yet we are not aware of any policy or operation directed against Germans in the USSR in this 

The effect size for the period of hostilities from 1933 to 1939 are much smaller compared to the war and post-war years. 
We get somewhat more conflicting results with regard to the statistical significance of these effects. 
For some years and specifications, the $p$-value is slightly smaller than 0.10, for others slightly greater. 
In any case, these results do not provide very strong evidence for a hypothesis that hostile  relations with a foreign country that are not accompanied by war substantially increase repressions of the respective minority. 
%This contradicts Mylonas'  \citeyearpar{mylonas_politics_2013} theory who suggests that geopolitical relations by themselves are the most important factor. 
%do not provide very strong evidence that 


%The effects are on the borders of 10\%  significance level. For some specifications, the $p$-values are slightly greater than 0.10 for other slightly smaller. 
%Furthermore, the effects are on the borders of sta
% For the period of hostilities without war from 1933 to 1939, we get conflicting results. 
% Whereas in the difference-in-differences we cannot reject the null hypothesis, our baseline synthetic control model implies positive and statistically significant effect. Nevertheless, even for the synthetic control, the estimated effect is fairly small corresponding to about 1\% rise in the repressions of Germans for the years 1933-1939.
 
This does not provide strong support for Mylonas'  \citeyearpar{mylonas_politics_2013} theory as the stark change in Soviet-German relations from 1933 to 1939 increased the repressions of Germans in the USSR only little if at all. 

Finally, increase in repressions is not limited to border areas nor is 
it  consistently higher there in comparison to the in-land. 
This suggests that the Soviet state did not use repressions mainly as a tool to secure its vulnerable border frontiers. %This contradicts the results of \citet{mcnamee_demographic_2019} who find that  


However, we also  have to be aware of the limitations of this study. 
%First, even though we analyzed data on over 2 million victims of repressions, 
First, the standard errors in our results might be slightly underestimated  since they do not take into account the uncertainty in the imputed values. 
Second, to correctly estimate the treatment effect, we have to assume that the treatment has no spillovers on the control units. Yet, it is fairly plausible that circumstances of war with Germany could increase repressions of other minorities as well.
%since they might be suspected of collaboration. 
Nonetheless, since we probably expect these spillovers  to be positive, our estimates would in that case be biased downward.
%the  missing data

%both methods provide some evidence to support this hypothesis, although the estimated effects are fairly small (around 2 \%). Moreover, there are several limitation to our study. The rise of Hitler might have made other non-German minorities less trustworthy in the view of the Soviet state as well because of the fear of collaboration with Germany in case of an invasion.
%We have seen that evidence for this hypothesis is rather weak. One possible explanation might be that the Germans were well represented in state institutions (including the NKVD) in regions of their heavy settlements and thus would not be prone to target their co-ethnics due to change in geopolitical relations. \citet[p. 126]{polian_against_2003} for example mentions that even on 31 June 1941, the Supreme Court of the Volga German ASSR sentenced a Russian $kolchoz$ chief for "delivering chauvinistic abuses against Germans residing in the USSR". The fruitful area for further research might be to compare how the rise in repressions differed for Germans living in areas with local autonomy (e.g. Volga German ASSR) and those living outside to see to what extent autonomy offered protection. 

% zmin se o limitacich - sutva, missing data
% zmin se o narustu v 1934 - nebylo centralne narizene - vysledek rozhodnuti lokalnich dustojniku NKVD