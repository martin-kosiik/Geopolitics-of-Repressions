We used difference-in-differences and the synthetic control method to test how changińg geopolitical relations between Soviet Union and Germany  affected repressions of Germans by the NKVD. 
Both methods provide evidence that the war significantly increased the arrests of Germans.
Specifically, the models estimate 2 to 3\% rise in the average of arrests during the war even though are some year-to-year deviation. 
Furthermore, we find that the increased repression persist almost undiminished for at least 10 year after the end of the war when the security concerns are no longer present.
This  suggests that use of violence by the state might be largely driven by out-group 
hostility rather than the strategic considerations emphasized in the literature. 
The strong and long persistence of the hostile attitudes after the war could potentially help explain the phenomenon of conflict trap (i.e. why violence tends to reoccur in the same places). 
% hod tam persitence of conlift, conflict trap
However, our methods do not enable us to determine the underlining mechanism which remains to be explored by further research. 

For the period of hostilities (but not war) from 1933 to 1939, we get conflicting results. Whereas in the difference-in-differences we cannot reject the null hypothesis, our baseline synthetic control model implies positive and statistically significant effect. Nevertheless, even for the synthetic control, the estimated effect is fairly small corresponding to about 1\% rise in the repressions of Germans for the years 1933-1939.


%both methods provide some evidence to support this hypothesis, although the estimated effects are fairly small (around 2 \%). Moreover, there are several limitation to our study. The rise of Hitler might have made other non-German minorities less trustworthy in the view of the Soviet state as well because of the fear of collaboration with Germany in case of an invasion.
%We have seen that evidence for this hypothesis is rather weak. One possible explanation might be that the Germans were well represented in state institutions (including the NKVD) in regions of their heavy settlements and thus would not be prone to target their co-ethnics due to change in geopolitical relations. \citet[p. 126]{polian_against_2003} for example mentions that even on 31 June 1941, the Supreme Court of the Volga German ASSR sentenced a Russian $kolchoz$ chief for "delivering chauvinistic abuses against Germans residing in the USSR". The fruitful area for further research might be to compare how the rise in repressions differed for Germans living in areas with local autonomy (e.g. Volga German ASSR) and those living outside to see to what extent autonomy offered protection. 

% zmin se o limitacich - sutva, missing data
% zmin se o narustu v 1934 - nebylo centralne narizene - vysledek rozhodnuti lokalnich dustojniku NKVD