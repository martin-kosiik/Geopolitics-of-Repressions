We used difference-in-differences and the synthetic control method to test whether the change in geopolitical relations between Soviet Union and Germany in 1933 caused the NKVD to target Soviet Germans more relative to other minority group. Both methods provide some evidence to support this hypothesis, although the estimated effects are fairly small (around 2 \%). Moreover, there are several limitation to our study. The rise of Hitler might have made other non-German minorities less trustworthy in the view of the Soviet state as well because of the fear of collaboration with Germany in case of an invasion.
%We have seen that evidence for this hypothesis is rather weak. One possible explanation might be that the Germans were well represented in state institutions (including the NKVD) in regions of their heavy settlements and thus would not be prone to target their co-ethnics due to change in geopolitical relations. \citet[p. 126]{polian_against_2003} for example mentions that even on 31 June 1941, the Supreme Court of the Volga German ASSR sentenced a Russian $kolchoz$ chief for "delivering chauvinistic abuses against Germans residing in the USSR". The fruitful area for further research might be to compare how the rise in repressions differed for Germans living in areas with local autonomy (e.g. Volga German ASSR) and those living outside to see to what extent autonomy offered protection. 

% zmin se o limitacich - sutva, missing data
% zmin se o narustu v 1934 - nebylo centralne narizene - vysledek rozhodnuti lokalnich dustojniku NKVD