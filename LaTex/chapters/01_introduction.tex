What determines the attitude of a state toward ethnic minorities within its borders? Why are some minorities accommodated or assimilated and others are politically excluded and repressed? Furthermore,  Why the position of a state toward its minorities changes in time? For example Soviet Union largely accommodated its minorities by in 1920s but it heavily repressed them in campaigns of mass terror 10 years later. 

%Some studies emphasize that certain domestic institution such as democracy can decrease the likelihood of persecutions. However this does not explain why the same states often treat its ethnic minorities so differently.  For example (give example, man)...\\
%\citet{mylonas_politics_2013} argues that geopolitical concerns play inportant role. Specifically, the state is likely to choose repression and exclusion against a minority group if a external power with ethnic ties to the group is a geopolitical enemy.  
\citet{mylonas_politics_2013} argues that geopolitical concerns play an important role. Specifically, state is likely to choose repression and exclusion if the ethnic minority's country of origin is seen as geopolitical enemy. The minority is then seen by the state as unreliable and  as potential fifth column in a case of conflict.  

We test this hypothesis on the case of German minority in Soviet union.
In 1933, Hitlers rise to power changed Germany from a neutral actor to ideological and geopolitical enemy in the perspective of the Soviet Union. We can then see how the repression changed before and after 1933 and compare it with other minorities. In particular, we use the individual arrests by soviet secret police (NKVD) as a dependent variable and employ the difference in difference strategy. 

%First, the Soviet Union was large multiethnic state whose attitude to its minorities drastically changed throughout the year with   

%We use the case of German minority in the Soviet Union to test this hypothesis for several reason.  

%We test hypothesis put forward by Mylonas (2012) according to which the host state is likely to choose repression and exclusion if the ethnic minority's country of origin is seen as geopolitical enemy. 


%We test this hypothesis on the case of German minority in Soviet using the rise of Hitler as a change of geopolitical relations. 
