\subsection{German–Soviet relations in the interwar period}
The relations between Weimar Germany and Soviet Union can be characterized as neutral or even cooperative. Both countries were somewhat isolated in the international system dominated by western powers (Great Britain, France, USA) and sought to find allies. The good relations were first established by the Treaty of Rappalo in 1922 in which both countries renounced the territorial and financial claims against the other and agreed to secret military cooperation \citep{gatzke_russo-german_1958} and then reaffirmed by the Treaty of Berlin in 1926. Furthermore, a trade treaty was signed between the two countries in 1925 \citep{morgan_political_1963}.

Hitler was named chancellor on 30 January 1933 and effectively become a dictator on 24 March 1933 by the passing of the Enabling Act. 
The relations with Soviet Union quickly turned hostile for several reasons.  First, Hitler called in \emph{Main Kampf} for Germany to obtain \emph{Lebensraum} (living space) in the east, presumably at the expense of the Soviet Union and he often spoke of Judeo-Bolsheviks. Moreover, Hitler soon after his rise to power banned the German Communist Party and started to persecute its members  \citep{haslam_soviet_1984}. 

Nevertheless, the Soviet Union and Nazi Germany were able to cooperate in areas of common interest under special circumstances in late 1930s. In particular in 1939 they sighed the  Molotov-Ribbentrop pact which guaranteed non-belligerence between Germany and the USSR and divided the spheres of influence in the Eastern Europe. This of course ended with German invasion into the USSR in June, 1941.  
Except the brief period of limited cooperation, the Germany represented an ideological and geopolitical opponent. The Soviet propaganda portrayed Nazi Germany as an existential enemy and rank-and-file NKVD officers would perceive it as such (which is why Molotov-Ribbentrop pact caused such a surprise). 
\subsection{Ethnic repressions in the Soviet Union}
In the 1920s, the Soviet policy towards its ethnic minorities was largely accommodating \citep{martin_affirmative_2001}. In some cases Autonomous Soviet Socialist Republics (ASSR) were established (including Volga German ASSR) which had given the regional minorities certain degree of independence. The languages and culture of minorities were even often promoted and minorities were encouraged to enter local goverments and party structures (so-called \emph{korenizatsiya} policy). 

This attitude changed drastically in the 1930s. First, the  \emph{korenizatsiya} policy started to be reversed. The Soviet state then gradually  began to target ethnic minorities for repressions  which culminated in the mass national operations of the NKVD of 1937-1938 resulting in more than 100 000 people being killed and many more sent to the Gulags (forced labor camps) \citep{snyder_bloodlands:_2011}. 
The persecutions  further escalated with the World War II. Following the German invasion into the Soviet Union in 1941, Stalin ordered deportation of about 400 000 Volga Germans into Kazakhstan and Siberia \citep{polian_against_2003}.