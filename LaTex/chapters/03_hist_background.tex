%\subsection{German minority in the Soviet Union}
In this section, we provide brief historical context for selected topics. Specifically, we first describe  changing geopolitical relations of Germany and the Soviet in the fist half of the 20th century. In the next subsection, we provide brief overview of the most important aspects of Soviet political repression. Finally, evolution of the Soviet policy towards its ethnic minorities is summarized.   
\subsection{German–Soviet Relations, 1921-1960} \label{subsec:relations}
The relations between Weimar Germany and Soviet Union can be characterized as neutral or even cooperative. Both countries were somewhat isolated in the international system dominated by the Western powers (Great Britain, France, USA) and sought to find allies. The good relations were first established by the Treaty of Rappalo in 1922 in which both countries renounced the territorial and financial claims against each other and agreed to secret military cooperation \citep{gatzke_russo-german_1958} and then reaffirmed by the Treaty of Berlin in 1926. Furthermore, a trade treaty was signed between the two countries in 1925 \citep{morgan_political_1963}.

Hitler was named chancellor on 30 January 1933 and effectively become a dictator on 24 March 1933 by the passing of the Enabling Act which gave him the power to enact laws without  approval of the parliament. 
The relations with Soviet Union quickly turned hostile for several reasons.  First, Hitler called in \emph{Main Kampf} for Germany to obtain \emph{Lebensraum} (living space) in the east, presumably at the expense of the Soviet Union and he often spoke of Judeo-Bolsheviks \citep[p. 6]{haslam_soviet_1984}. Moreover, Hitler's anti-communist was one of factors contributing to his political success as he presented himself as the only leader strong enough to  prevent a Communist revolution in Germany. This was not only empty rhetoric as he soon after his rise to power banned the German Communist Party and started to persecute its members \citep[chapter 5]{evans_coming_2004}. 

The opposition to fascism led to change in policy of the Communist International (Comintern)  with appointment Georgi Dimitrov as  its general secretary in 1934. The Communist parties in democratic countries were now encouraged to form coalitions (Popular Fronts) with social democratic parties to prevent rise of fascism, in contrast to the previous aggressive and uncompromising approach. This policy was affirmed by the Seventh World Congress of the Comintern in 1935 \citep{haslam_comintern_1979}.

The newly formed Popular Front coalitions won elections and entered government in some European countries including France and Spain. In Spain however, the coup of nationalists against the new government in 1936 sparked a civil war. The Soviet Union heavily supported the republican government, while Germany supplied the nationalists which further increased the tensions between the two countries. 
As a response to increasing tensions, Japan and Germany signed the Anti-Comintern Pact in 1936 in which they committed to  co-operate for defense against communistic disintegration. 
%Meanwhile in the Soviet Union, many people were persecuted for alleged cooperation with Germany including leading general Mikhail Tukhachevsky. 

The orientation of German foreign policy began to shift in spring of 1939.
Until that point, Hitler hoped that he could ally with Poland in a war against the Soviet Union or that Poland would at least allow the passing of German troops \citep[chapter 26]{weinberg_hitlers_2010}. 
%tried to court Poland to join the Anti-Comintern Pact against the Soviet Union 
But Poland repeatedly refused the German offers for closer relations such as to join the Anti-Comintern Pact and thus Hitler changed the strategy and  in April 1939 ordered the German army to began planing for the invasion of Poland \citep[p. 621]{kotkin_stalin:_2017}. However, France and Great Britain granted security guarantees to Poland in March 1939. 
Hitler thus tried to negotiate neutrality of the Soviet Union in war to avoid simultaneously facing  Western powers, Poland and the Soviet Union.
Soviet neutrality was potentially beneficial for Stalin too. A long and 
costly war would weaken the both the capitalist and fascist enemies of the 
Soviet Union. Moreover, Stalin believed that conditions of war could bring 
about socialist revolutions in those countries just as in Russia in 1917.  
After brief negotiations, on 23 August 1939   the  Molotov-Ribbentrop pact 
was signed between Germany and the USSR which guaranteed non-belligerence 
between the two countries. In addition,  a secret protocol of the treaty marked the German and Soviet spheres of influence in Eastern Europe.

%Therefore in case of war, Germany could be facing coalition of Western powers, Poland and the Soviet Union. 


%Nevertheless, the Soviet Union and Nazi Germany were able to cooperate inareas of common interest under special circumstances in late 1930s. Inparticular, on 23 August 1939 they sighed the  Molotov-Ribbentrop pact which guaranteed non-belligerence between Germany and the USSR and dividedthe spheres of influence in the Eastern Europe. This of course ended with German invasion into the USSR in June, 1941.  

%Except the brief period of limited cooperation, the Germany represented an ideological and geopolitical opponent. The Soviet propaganda portrayed Nazi
%Germany as an existential enemy and rank-and-file NKVD officers would
%perceive it as such (which is why Molotov-Ribbentrop pact caused such a surprise). \citep{kotkin_stalin:_2017}
The pact of the two former ideological enemies caused great shock and astonishment both among Party officials and ordinary people. 
Victor  \citet[p. 332]{kravchenko_i_1947}, a Soviet official who later defected to the US,   described in  his memoir the disbelief upon hearing about the pact  
\begin{quote}
There must be some mistake, I thought, and everyone around me seemed equally incredulous. After all, hatred of Nazism had been drummed into our minds year after year.  The big treason trials [...] have rested on assumption that Nazi Germany and its Axis friends [...]  were preparing to attack us. 
\end{quote}
Another party official later recalled that \enquote{it left us all stunned, bewildered, and groggy with disbelief} \citep[p. 137]{robinson_black_1988}. 

Nazi Germany attacked Poland on 1 September 1939 from the west and shortly after that, on 17 September, the Red Army invaded the eastern part of the country.  As was agreed in the pact, Poland was partitioned between Germany and the Soviet Union.
However, the mistrust between the two countries was still present as evidenced by a violent clash of German and Soviet troops near Lwów on 20 September \citep[p. 685]{kotkin_stalin:_2017}
%The pact, however, did last only 2 years and was marked by strong mutual distrust. 
%Other 

Hitler enjoyed major success in the first years of the war.
By summer 1940, German forces defeated French army and annexed Denmark and Norway. However, German industry was severely lacking raw materials needed in war effort against Britain and the US which, according to some historians, motivated Hitler to invade the resource-rich USSR \citep{tooze_wages_2008}. The German attack on the Soviet Union on 22 June 1941 ended 2 years of fragile cooperation.
Stalin did not anticipate the invasion despite the warnings 
Although Stalin received numerous warnings by his intelligence about the impeding German attack, he was generally dismissive of them as British efforts to embroil him in war with Germany  \citep[chapter 14]{kotkin_stalin:_2017}. 

The Eastern Front became the bloodiest theater of World War II with more than 10 million soldiers killed in combat and another 3.3 million of Soviet prisoners of war starved to death  by Germans \citep[p. 155]{snyder_bloodlands:_2011}. Moreover, the Eastern Front was site of the worst atrocities committed on the civilian population, most notably the Holocaust. 
%during  the war It became the site of the worst atrocities including the Holocaust, starving of more than 3.3 million of Soviet prisoners of war and other. 

After the surrender of Germany in May 1945 its territory was partitioned into 4 occupation zones.  Various industrial disarmament programs were put in place in all occupation zones to limit and control the German military capacity. Thus, in the post-war period militarily weak Germany no longer presented a geopolitical threat as it did before.
Instead,  the rivalry  of the Soviet Union and the United States became  the new main source of tensions in the international relations. 

To summarize, there were several events in the period from 1921 to 1960 that fundamentally  altered the Soviet-German relations. First,  Hitler's rise to power, which was definitely consolidated by the passing of The Enabling Act on 23 March 1933, brought in heightened hostilities and tensions into the Soviet-German relations.
Another turning point was the Molotov–Ribbentrop Pact signed 23 August 1939 which started a brief period of limited cooperation between the two countries. 
On 22 June 1941, the German invasion of the Soviet Union officially terminated the pact  marking the beginning of one of the most bloody conflicts of World War II. 
 The war finally ended on  8 May 1945  with unconditional surrender of Germany. 
\subsection{Soviet Political Repressions}
The Soviet Union had massive coercive apparatus. The Soviet secret police (which was throughout the years named the Cheka, OGPU, NKVD, MVD and  the KGB)\footnote{We will refer to the Soviet secret police as the NKVD in this text since this was the name of the agency for the largest part of the period of our interest} employed at its height (1937–1938)  270,730 persons \citep[p. 2]{gregory_terror_2009}. 
%The gulag  
The political repressions were usually carried under Article 58 of the Criminal Code. The Article 58 punished counter-revolutionary activities 
%defined as "any action aimed at overthrowing, undermining or weakening of the power of workers' and peasants' Soviets... and governments of the USSR". This 
which included treason, espionage, counterrevolutionary propaganda or agitation and  failure to report any of these crimes.  
In practice, this broad definition meant that anyone regarded as politically inconvenient could be arrested and prosecuted. 

During the mass operations, the central office of the NKVD would typically set quotas for the number of arrests which the regional branches were supposed to reach and exceed \citep[chapter 6]{gregory_terror_2009}. The local NKVD officer had to decide themselves who to target to meet the quotas. 

The sentences were in most cases issued extrajudicially by so-called \enquote{troikas}, three-person committees composed of a regional NKVD chief, a regional party leader, and a regional prosecutor. The NKVD chief usually dominated the process as party leaders sometimes feared that they themselves would be targeted \citep[p. 82]{snyder_bloodlands:_2011}. Only rarely was a person acquitted from his charge. 
% mozna napis neco o vyslichani jak je ve snyderovy na 82
The most common sentences for political crimes in the Stalinist period were  execution and prison term in a labor camp (Gulag) \citep[p. 21]{gregory_terror_2009}. 
A term in the Gulag of less then 5 years was considered lighter sentence in these cases.
% napis tady o Gulagech 

 With the rise in  repressions in the 1930s, the Gulag system significantly expanded. At its height, it consisted of at least 476 distinct camp complexes each containing hundreds of prisoners. The Gulag system offered the Soviet state cheap source of labor that produced substantial amount the country's coal, timber, and gold supply. The mortality of prisoners was high due to heavy work, malnutrition, and cold climate \citep{applebaum_gulag:_2003}. 

The death of Stalin in 1953 marked a start of decline in political repressions in the USSR. The new Soviet leader, Nikita Khrushchev, denounced Stalin and the mass repressions of his period in his speech \emph{On the Cult of Personality and Its Consequences} in 1956. The suppression of dissent continued in the Khrushchev and  Brezhnev era but in much milder form. Khrushchev gradually dismantled the Gulag system, granted amnesty to many political prisoners and started the process of rehabilitation of victims of the Stalinist period although they were limited to only some categories of victims and offences \citep{applebaum_gulag:_2003, dobson_khrushchevs_2009}. 

\subsection{Ethnic Minorities in the USSR}
The Soviet Union was from its inception a multi-ethnic state. 
According to the 1926 Census, the Russians made up only half of the total population.\footnote{The census data were obtained  Institute of Demography of the National Research University Higher School of Economics} Among other large ethnic group were Ukrainians, Belorussians and Kazakhs.  A significant fraction of citizens of the USSR belonged to ethnic groups with their own independent states including Polish, German, Estonian, Latvian, Lithuanian, Finish,  and Greek minorities. 
The Bolshevik elites were aware of the multi-ethnic nature of their newly formed state and wanted to  avoid a perception of the Soviet Union as a project of Russian imperialism. Furthermore, the Bolsheviks hoped that
they could exert political influence in countries with cross-border ethnic  ties to Soviet diaspora nationalities 
by promoting the interests of minorities in the USSR (this was know as the \enquote{Piedmont Principle}).

As a consequence, the Soviet policy towards its ethnic minorities in the 1920s was largely accommodating \citep{martin_affirmative_2001}. The languages and culture of minorities were promoted and minorities were encouraged to enter local governments and party structures (so-called \emph{korenizatsiya} policy). Some minority groups were well represented even in the NKVD \citep[p. 25]{gregory_terror_2009}.
In some cases Autonomous Soviet Socialist Republics (ASSR) were established (including Volga German ASSR) which had given the regional minorities certain degree of independence. 
%One of the rationale for \emph{korenizatsiya} policy given by Lenin was so-called Piedmont principle. The Bolshevik elite hoped that by promoting the interests of minorities in the USSR, 
%they could project influence in other countries by explointing the cross-border ethnic ties of minorities. 
%the ethnic minorities would use their cross-border ties to start socialist revolutions. 

This attitude changed drastically in the 1930s.
First, the  \emph{korenizatsiya} policy started to be reversed in the 1932.
From 1934, the NKVD started to deport ethnic minorities from the state frontier zone in Eastern Europe. This involved forced resettlement of  30 000 of Ingermanland Finns and tens of thousands of Poles and Germans to  Kazakhstan and West Siberia  \citep[p. 95]{polian_against_2003}.
% Mozna zmin neco o deportacich korejcu a cinanu
%681,692 mortal victims of the Great Terror were Polish. 
In 1937 and 1938, the NKVD conducted mass operations specifically
targeted at minorities with cross-border ethnic ties. 
Poles, Latvians, Germans, Estonians, Finns, Greeks, Chinese, and Romanians were arrested in large numbers as supposed spies and saboteurs of foreign governments.
More than 320 000 people were arrested in the national operation out of which about 250 000 were executed \citep[p. 855]{martin_origins_1998}. %The Polish operation claimed the highest number of victims (110 000 deaths) but many . %snyder bloodlands
%The Soviet state then gradually  began to target ethnic minorities for repressions  which culminated in the mass national operations of the NKVD of 1937-1938 resulting in more than 100 000 people being killed and many more sent to the Gulags \citep{martin_origins_1998, snyder_bloodlands:_2011}. 

The persecutions  further escalated with the World War II. Following the German invasion into the Soviet Union in 1941, Stalin ordered deportation of about 430 000 Soviet Germans (most of them living in Volga German ASSR)  into Kazakhstan and Siberia \citep[p. 134]{polian_against_2003}. Similar \enquote{preventive} deportation followed for  Finns and Greeks as well. Between 1943-1944, forced resettlement of another six ethnic groups (Karachais, Kalmyks, Chechens, Ingushetians, Balkars, and Crimean Tatars) were carried out for alleged or actual cooperation of some of these minorities with the German troops (even if many more served in the Red Army). 

% in the post-war period
% dopln informaci o mortalite (koukni do Bugaie 1996 nebo Snydera)
% pridej informaci o povalecnych deportacich 
% pridej informaci o rehabilitaci