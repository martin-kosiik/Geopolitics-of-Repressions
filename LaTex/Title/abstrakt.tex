\pagestyle{empty}

\section*{Bibliographic note}

\noindent KOSÍK, Martin. \textit{The Geopolitics of Repressions}.  2019. x p.
Bachelor thesis. Charles University, Faculty of Social Sciences, Institute of Economic Studies. Thesis Supervisor: doc. PhDr. Julie Chytilová, Ph.D. \\
%\textbf{\textcolor{red}{[p?i tvorb? bibliografick?ch citac?
%vych?zejte z normy ?SN ISO 690, ?SN ISO 690-2 nebo podle jin?ch
%doporu?en?ch cita?n?ch styl?]}}\\

\section*{Abstract}
  This thesis studies how geopolitical concerns influence attitudes of a state toward its ethnic minorities.
  %I exploit the  Hitler's rise to power in 1933 as an exogenous shock to Soviet-German relations. 
    Using  data digitized from archival sources on  more than 2 million individual arrests by the Soviet secret police, I apply difference-in-differences and synthetic control method to estimate how changing geopolitical relations influenced repressions of Germans in the USSR. 
   % We find that repressions in the period 
    The results of both methods imply that that the arrests of Germans relative to other minorities increased by approximately 2\% to 4\% following the German invasion into the Soviet Union in 1941. 
    Furthermore, the impact of war  appears to be highly persistent since there is virtually  no decline in the estimated effect on repressions for nearly 10 years after the end of war.
    %These finding are robust to different sensitivity checks. 

\section*{Keywords}
repression, geopolitics, Soviet Union, difference-in-differences, synthetic control method, \\
\newpage
\section*{Abstrakt}
Tato práce zkoumá jak geopolitické zájmy ovlivňují vzah státu k jeho etnickým minoritám. 
Aplikuji metodu rozdílu v rozdílech (difference-in-differences) a metodu syntetické kontroly (synthetic control method) s použitím  dat digitalizovaných z archivních zdrojů obsahujích více než 2 miliony záznamů individuálních zatčení sovětskou tajnou policií.  
Výsledky obou metod implikují, že 
Vliv války  se zdá být velmi perzistentní, jelikož nepozorujeme téměř  žádný pokles odhadovaného effektu na represe skoro 10 let po ukončení války. 

\section*{Klíčová slova}
represe, geopolitika, Sovětský svaz, difference-in-differences, synthetic control method\\

\newpage
