\documentclass[12pt]{article}
\usepackage[utf8]{inputenc}

\title{The Geopolitics of Persecutioins \\
  \large The Case of German Minority in the Soviet Union}
\author{Martin Kosík}
\date{December 2018}

\usepackage{natbib}
\usepackage{graphicx}
\usepackage{amsmath}
\usepackage{enumitem}


\begin{document}

\maketitle


\section{Introduction}
Why are some ethnic minorities excluded and persecuted by the state whereas other are given autonomy and accommodation? Some studies emphasize that certain domestic institution such as democracy can decrease the likelihood of persecutions. However this does not explain why the same states often treat its ethnic minorities so differently.  For example (give example, man)...\\
We test hypothesis put forward by Mylonas (2012) according to which the host state is likely to choose repression and exclusion if the ethnic minority's country of origin is seen as geopolitical enemy. 

We test this hypothesis on the case of German minority in Soviet using the rise of Hitler as a change of geopolitical relations. 


\begin{figure}[h!]
\centering
\includegraphics[scale=1.7]{}
\caption{The Universe}
\label{fig:universe}
\end{figure}
\section{Literature review}

\section{Historical Background}
\subsection{German minority in the Soviet Union}
\section{Data}
We use declassified archival materials on the victims of political repressions in the Soviet Union digitized by Memorial and aggregated by Zhukov (). This dataset includes 2.6 individual arrest by the Soviet secret police (NKVD) between  the years 1921 and 1959.
\section{Methodology}
We follow the standard difference-in-difference method
$$ \log\left(y_{it}\right) = \beta_0 + \beta_1 \: german + \delta_t + \beta_3 german \cdot post + u_{it} $$

\section{Results}

\section{Conclusion}
``I always thought something was fundamentally wrong with the universe'' \citep{adams1995hitchhiker}



\bibliographystyle{plain}
\bibliography{references}
\end{document}
