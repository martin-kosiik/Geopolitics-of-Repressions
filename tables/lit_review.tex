There is large literature documenting both higher desire for and  investment in sons in  developing countries. Cultural norms and practices such as patrilineality, patrilocality, dowry and religious rituals with special role of sons have been suggested as an explanation for different degrees of son preference across countries  \citep{jayachandran_roots_2015}.

Fewer studies have tried to empirically estimate the causal effect of these practices. \citet{ebenstein_patrilocality_2014} finds strong positive correlation between sex ratios and rates  of coresidence  of adult sons and elderly parents both between within and across countries. 
In another study,  \citet{ebenstein_son_2010} examine the effects of voluntary pension program in rural China. They
show that higher availability of the program on the county-level was associated with a slower rise in sex ratios.  Moreover, parents without a son were more likely to participate in the program. 
This provides some evidence that old-age support from sons in patrilocal societies is an important reason for son preference. 
However, \citet{almond_o_2009} observe skewed sex ratios   even among  Asian immigrants to Canada  which cannot be explained by old-age support alone. 


\citet{jayachandran_why_2017} show that favoritism toward eldest sons is the main cause of steeper birth order gradient in height in India. 
Moreover, they find that the birth order gradient  is flatter in  Indian states that are traditionally matrilineal. The gradient is also shallower for Indian Muslims in comparison with Hindus which might be a consequence of special role of sons in Hindu rituals. 


\citet{rossi_gender_2015} study duration of birth intervals to estimate variation in gender preferences in Africa. They find strong son preference in North Africa and somewhat weaker son preference in Mali, Senegal, and in the Great Lakes region (see figure \ref{son}). Furthermore, there is no difference in son preference between Muslims and members of other religions. However,  they observe that matrilineal ethnic groups exhibit significantly lower son preference even though the study does not establish a causal connection. 