\begin{table}[t]

\caption{\label{tab:sc_weights}Pre-treatment characteristics of ethnic groups in the USSR for SC}
\centering
\begin{threeparttable}
\begin{tabular}{lrrrr}
\toprule
Ethnic group & Total population & Ling. similarity to Russian & Urbanization rate & Ind. country\\
\midrule
Armenian & 1 567 568 & 1 & 35.45 & 0\\
Belorussian & 4 738 923 & 4 & 10.32 & 0\\
Estonian & 154 666 & 0 & 23.00 & 1\\
German & 1 238 549 & 1 & 14.92 & 1\\
Greek & 213 765 & 1 & 21.21 & 1\\
Chechen & 318 522 & 0 & 0.98 & 0\\
Chinese & 10 247 & 0 & 64.87 & 1\\
Jewish & 2 599 973 & 1 & 82.43 & 0\\
Kabardian & 139 925 & 0 & 1.27 & 0\\
Kalmyk & 129 321 & 0 & 1.29 & 0\\
Korean & 86 999 & 0 & 10.52 & 0\\
Latvian & 141 703 & 2 & 42.31 & 1\\
Lithuanian & 41 463 & 2 & 63.16 & 1\\
Ossetian & 272 272 & 1 & 7.86 & 0\\
Polish & 782 334 & 3 & 32.75 & 1\\
Tatar & 2 916 536 & 0 & 15.48 & 0\\
Ukrainian & 31 194 976 & 4 & 10.54 & 0\\
Altai & 39 062 & 0 & 0.30 & 0\\
Balkar & 33 307 & 0 & 1.23 & 0\\
Bashkir & 713 693 & 0 & 2.12 & 0\\
Bulgarian & 111 296 & 3 & 6.26 & 0\\
Buryat & 237 501 & 0 & 1.05 & 0\\
Finnish & 134 701 & 0 & 10.55 & 1\\
Georgian & 1 821 184 & 0 & 16.93 & 0\\
Hungarian & 5 476 & 0 & 63.33 & 1\\
Chuvash & 1 117 419 & 0 & 1.60 & 0\\
Japanese & 93 & 0 & 76.34 & 1\\
Karelian & 248 120 & 0 & 2.91 & 0\\
Kazakh & 3 968 289 & 0 & 2.18 & 0\\
Khakas & 45 608 & 0 & 1.08 & 0\\
Komi & 375 871 & 0 & 2.56 & 0\\
Mari & 428 192 & 0 & 0.84 & 0\\
Moldovan & 278 905 & 1 & 4.86 & 0\\
Mordvin & 1 340 415 & 0 & 2.19 & 0\\
Russian & 77 791 124 & 5 & 21.32 & 1\\
Udmurt & 504 187 & 0 & 1.21 & 0\\
Uzbek & 3 904 622 & 0 & 18.66 & 0\\
Yakut & 240 709 & 0 & 2.20 & 0\\
\bottomrule
\end{tabular}
\begin{tablenotes}
\item \textit{Note: } 
\item Total population and urbanization rate of the ethnic group in the USSR is taken from 1926 census. The linguistic similarity to Russian is measured by the number of common nodes in the language tree (cladistic similarity). Independent country equals one if the ethnic group was a core group in an independent state that existed in the interwar period.
\end{tablenotes}
\end{threeparttable}
\end{table}